\documentclass{article}
\usepackage[utf8]{inputenc}
\usepackage{graphicx}


\title{Central Saturns}
\author{Basmala Kamal,
Ahmed Moussa,
Sama Yasser}

\date{19-01} 
\begin{document}

\maketitle

\section{Introduction}
The universe is full of mysteries and discoveries that have not been revealed yet. Massive celestial bodies to the unicellular living organisms. Scientists work hard to reveal these secrets and understand the physical background of bodies no matter living organisms, objects, planets, or galaxies and see what features are common between them. One of these features that have been discovered is the “center of mass” where it’s the point of balance in anybody.
 
 All bodies have a center of mass but a system of bodies as a whole can also have their own center of mass. This is also applied to the solar system where the sun and the eight planets as a whole have a barycenter that changes its position from time to time according to the position of planets. It’s hard to find the position of the barycenter due to this variation and the complicated equations so it takes a long time to get the final result and determine the position. That’s why scientists count on programming to do these complicated equations in just some seconds with less effort.

 This project aimed to find the position of the barycenter in any time interval in just some seconds. A program was made using PYTHON programming language with VPYTHON (PYTHON plus a 3D graphics module called visual) to graph the orientation of planets and define the position of the barycenter in the time interval the user decides. Some mathematical equations were used in this in addition to some constants.





\section{Methodology}
  First of all, it is a must to determine libraries used in a specific program to identify what is going to be achieved, as well as the constants for any scientific program that is calculation-based. The program is asked to take an input from the user as two separated dates. Furthermore, the two dates are going to be involved in a calculation to convert its form from text “string” into date format to calculate the period of in-between time. This period is the simulated time done by the program. 
  Then, by using object-oriented programming to assign general properties for planets, every single property specified for each planet takes place to identify one another. For instance, the properties needed might be the mass and radius. 
 Eventually, it is all about the shaping. How all planets need to have its texture and position among the solar system objects. Afterwards, it is time for the program to function and output the barycenter of the solar system where every planet is in the right place according to dates. All solar system objects rotate and differentiate in a particular rate until the determined period of time is over. 



       
\section{Results}
Firstly, the program had a lot of errors including non-uniformity of the orbits, errors in displaying the position of the center of mass of the solar system and running period errors. 
After countless tries to fix all the errors in the project and by revising the concepts of astrophysics and python programming language, and editing the code, it ran flawlessly.
A 3D simulation of the solar system was displayed after running the code showingٍ the planets orbits, their orbital speeds relative to each other and the position of the center of mass. It was found that the position of the center of mass of the solar system changes slightly as a result of the motion of planets. Jupiter and Saturn had the strongest effect. The center of mass was sometimes placed outside the sun’s surface and the sun completes a rotation relative to this point in about 11 to 12 years. \\
\includegraphics{1.png}
\includegraphics{2.png}

\section{Analysis}
The scientific base that supported our program was mainly the studies behind center of mass that resulted to determination of the barycenter  of the solar system. Besides, it was a must to understand and analyze the universal law of gravitation while
dealing with the solar system changes over time especially while regarding its
barycenter. Furthermore, calculating orbital velocity is due to simulate the different positions where planets could be in a particular period of time. It would also change according to the factor of the simulated time.
The project simulates the idea of the topic which is to find the center of mass of the solar system. The barycenter position is not constant and changes from time interval to another. So, it makes it difficult for the human brain to determine the position of it, and it will take much time and consume effort. That’s why the program was established, in order to use the equations and some constant values to determine the position of the barycenter in the time interval the user decide in just seconds instead of lots of effort and time. 
  The program was based on some constants and mathematical equations and supported with scientific laws. The mass of the sun and the eight planets were added, the radii of each planet and the sun were provided, the gravitation constant “G” where added with its value. All these values were added in order to obtain the barycenter by three equations which are: the law of gravitation and orbital velocity. 
The orbital velocity =  
$ \sqrt\frac{G.M}{r}$

where, it uses the value of “G” constant, mass, and radius of the sun or planet.
The center of mass of the whole system = 
   $X_c = \frac{m_1x_1+m_2x_2..}{m_1+m_2..}$
   where $m_n$ is the mass of planets times $X_n$ which is the position of the planet. 

The law of gravitation: $F = \frac{Gm_1m_2}{r^2}$

All these equations combined with the time interval the user determines, the program will determine the barycenter of the solar system. 

\section{Role Division}
All the work was distributed evenly between us without any load on other members. In some circumstances one of the members can’t complete the task he has because of important excuses so; he compensates for that and works more in the following tasks. 
1.	Member (1): Basmalah Kamal: Changed the abstract in latex format in the first phase, wrote part of the codes in the program, presented half of the presentation in the evaluation online meeting with the head, wrote the methodology and point of the analysis, changed the report to latex format and participated in the video in the fourth phase. 

2.	Member (2): Sama Yasser: wrote the abstract in the first phase, made the flowchart in the second phase, presented the other half of the presentation in the evaluation online meeting with the head, wrote the introduction, the role division, and two points of the analysis and participated in the video in the fourth phase. 

3.	Member (3): Ahmed Moussa: wrote the other part of the codes of the program in the third phase, wrote the results, made the references in the IEEE format in the fourth phase.

\section{References}
\emph{
1- Palen, S., 2011. Schaum's Outline Of Theory And Problems Of Astronomy. New York: McGraw Hill Professional.
\newline
2- Freedman, R. and Kaufmann, W., 2008. Universe. New York, NY: W.H. Freeman and Co.
\newline
3- Matthes, E., 2016. Python Crash Course. 
\newline
4- Serway, R., Vuille, C. and Hughes, J., n.d. College Physics. }


\end{document}
